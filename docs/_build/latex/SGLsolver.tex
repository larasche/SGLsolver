%% Generated by Sphinx.
\def\sphinxdocclass{report}
\documentclass[letterpaper,10pt,english]{sphinxmanual}
\ifdefined\pdfpxdimen
   \let\sphinxpxdimen\pdfpxdimen\else\newdimen\sphinxpxdimen
\fi \sphinxpxdimen=.75bp\relax

\PassOptionsToPackage{warn}{textcomp}
\usepackage[utf8]{inputenc}
\ifdefined\DeclareUnicodeCharacter
% support both utf8 and utf8x syntaxes
\edef\sphinxdqmaybe{\ifdefined\DeclareUnicodeCharacterAsOptional\string"\fi}
  \DeclareUnicodeCharacter{\sphinxdqmaybe00A0}{\nobreakspace}
  \DeclareUnicodeCharacter{\sphinxdqmaybe2500}{\sphinxunichar{2500}}
  \DeclareUnicodeCharacter{\sphinxdqmaybe2502}{\sphinxunichar{2502}}
  \DeclareUnicodeCharacter{\sphinxdqmaybe2514}{\sphinxunichar{2514}}
  \DeclareUnicodeCharacter{\sphinxdqmaybe251C}{\sphinxunichar{251C}}
  \DeclareUnicodeCharacter{\sphinxdqmaybe2572}{\textbackslash}
\fi
\usepackage{cmap}
\usepackage[T1]{fontenc}
\usepackage{amsmath,amssymb,amstext}
\usepackage{babel}
\usepackage{times}
\usepackage[Bjarne]{fncychap}
\usepackage{sphinx}

\fvset{fontsize=\small}
\usepackage{geometry}

% Include hyperref last.
\usepackage{hyperref}
% Fix anchor placement for figures with captions.
\usepackage{hypcap}% it must be loaded after hyperref.
% Set up styles of URL: it should be placed after hyperref.
\urlstyle{same}
\addto\captionsenglish{\renewcommand{\contentsname}{Contents:}}

\addto\captionsenglish{\renewcommand{\figurename}{Fig.\@ }}
\makeatletter
\def\fnum@figure{\figurename\thefigure{}}
\makeatother
\addto\captionsenglish{\renewcommand{\tablename}{Table }}
\makeatletter
\def\fnum@table{\tablename\thetable{}}
\makeatother
\addto\captionsenglish{\renewcommand{\literalblockname}{Listing}}

\addto\captionsenglish{\renewcommand{\literalblockcontinuedname}{continued from previous page}}
\addto\captionsenglish{\renewcommand{\literalblockcontinuesname}{continues on next page}}
\addto\captionsenglish{\renewcommand{\sphinxnonalphabeticalgroupname}{Non-alphabetical}}
\addto\captionsenglish{\renewcommand{\sphinxsymbolsname}{Symbols}}
\addto\captionsenglish{\renewcommand{\sphinxnumbersname}{Numbers}}

\addto\extrasenglish{\def\pageautorefname{page}}

\setcounter{tocdepth}{1}



\title{SGLsolver Documentation}
\date{Oct 03, 2020}
\release{0.1}
\author{Lara Aschenbeck und Malte Kehlenbeck}
\newcommand{\sphinxlogo}{\vbox{}}
\renewcommand{\releasename}{Release}
\makeindex
\begin{document}

\pagestyle{empty}
\sphinxmaketitle
\pagestyle{plain}
\sphinxtableofcontents
\pagestyle{normal}
\phantomsection\label{\detokenize{index::doc}}



\chapter{SGLSOLVER}
\label{\detokenize{README:sglsolver}}\label{\detokenize{README::doc}}

\section{Introduction}
\label{\detokenize{README:introduction}}
Package containing routines for solving the schrödinger equation for
different potentials.


\chapter{Usage}
\label{\detokenize{README:usage}}

\section{Input}
\label{\detokenize{README:input}}
The following text shows an example how the .int data has to be structured

schrodinger.int
\begin{quote}

2.0 \# mass
-2.0 2.0 1999 \# xMin xMax nPoint
1 5 \# first and last eigenvalue to print
linear \# interpolation type
2 \# nr. of interpolation points and xy declarations
-2.0 0.0
2.0 0.0
\end{quote}


\section{Apllication}
\label{\detokenize{README:apllication}}

\subsection{Solvers}
\label{\detokenize{README:solvers}}\begin{description}
\item[{./solvers.py -d {[}Directory{]}}] \leavevmode
Solves the SGL for the given problem in the given Directory

\end{description}

Returns:
\begin{itemize}
\item {} 
energies.dat: .dat containing energie and eigenvalues

\item {} 
potential.dat:        .dat containing the interpolatet potential

\item {} 
wavefunction.dat:     .dat containing the eigenvectors

\end{itemize}


\subsection{Plotmain}
\label{\detokenize{README:plotmain}}\begin{description}
\item[{./plotmain.py -d {[}Directory {]} -ymin {[}Ymin{]} -ymax {[}Ymax{]} -s {[}Scaling{]}}] \leavevmode
Visualise the solved problem in a graph

\end{description}

Returns:
\begin{itemize}
\item {} 
graph.pdf:            .pdf containing graphs

\end{itemize}


\chapter{Indices and tables}
\label{\detokenize{index:indices-and-tables}}\begin{itemize}
\item {} 
\DUrole{xref,std,std-ref}{genindex}

\item {} 
\DUrole{xref,std,std-ref}{modindex}

\item {} 
\DUrole{xref,std,std-ref}{search}

\end{itemize}



\renewcommand{\indexname}{Index}
\printindex
\end{document}